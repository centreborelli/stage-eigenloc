\documentclass[a4paper,11pt]{article}
\usepackage[utf8]{inputenc}
\usepackage[french]{babel}
\usepackage{amsmath}
\usepackage[osf,sups]{Baskervaldx} % lining figures
\usepackage[bigdelims,cmintegrals,vvarbb,baskervaldx,frenchmath]{newtxmath} % math font
%\usepackage[cal=boondoxo]{mathalfa} % mathcal from STIX, unslanted a bit
%\usepackage{float}
\usepackage{graphicx}
\usepackage{url,hyperref}
%\usepackage{color}

\setlength{\parindent}{0em}
\setlength{\parskip}{1em}

\addtolength{\hoffset}{-2em}
\addtolength{\textwidth}{4em}
%\addtolength{\voffset}{-5em}
%\addtolength{\textheight}{10em}

\addtolength{\voffset}{-6em}
\addtolength{\textheight}{12em}

% coco's macros
\def\R{\mathbf{R}}
\def\F{\mathcal{F}}
\def\x{\mathbf{x}}
\def\y{\mathbf{y}}
\def\u{\mathbf{u}}
\def\Z{\mathbf{Z}}
\def\d{\mathrm{d}}
\DeclareMathOperator*{\argmin}{arg\,min}
\DeclareMathOperator*{\argmax}{arg\,max}
\newcommand{\reference}[1] {{\scriptsize \color{gray}  #1 }}
\newcommand{\referencep}[1] {{\tiny \color{gray}  #1 }}
\newcommand{\unit}[1] {{\tiny \color{gray}  #1 }}

\begin{document}
\thispagestyle{empty}

\begin{center}
	\Large Différentiabilité des éléments propres
\end{center}

On identifie toujours~$\R^3$ avec l'ensemble de matrices réelles symétriques~$2\times
2$, par la correspondance
\[
	\begin{pmatrix}
		x\\
		y\\
		z\\
	\end{pmatrix}
	\longleftrightarrow
	\begin{pmatrix}
		x & y \\
		y & z \\
	\end{pmatrix}
\]
De façon similaire, on identifie~$\R^{\frac{N(N+1)}{2}}$
avec les matrices symétriques~$N\times N$.

{\bf Exercice 1.}
Soit~$S^+$ l'ensemble de matrices~$2\times 2$ symétriques réelles semi-définies
positives, et~$S^+_*$ le sous-ensemble de~$S^+$ formé par les matrices dont les
deux valeurs propres sont différentes.  Démontrez que ces deux ensembles sont
des ouverts de~$\R^3$ et décrivez-les géométriquement.

{\bf Exercice 2.}
Pour~$k=1,2$, soit~$\lambda_k:S^+_*\to\R$ l'application que à chaque matrice
lui assigne sa~$k$-ième valeur propre, ordonnées par ordre croissant.
Démontrez que~$\lambda_k$ est bien définie,
écrivez
explicitement~$\lambda_1(x,y,z)$ et~$\lambda_2(x,y,z)$, et démontrez que les
fonctions~$\lambda_k$ sont~$\mathcal{C}^\infty$ dans leur domaine de
définition.

{\bf Exercice 3.} Est-il possible d'étendre les fonctions~$\lambda_k$ sur
tout~$S^+$ de façon qu'elles soient toujours~$\mathcal{C}^\infty$?
Et~$\mathcal{C}^0$ ?

{\bf Exercice 4.}
Démontrez qu'il est possible de définir deux fonctions~$\mu_k:S^+_*\to\R^2$
avec les propriétés suivantes:\\
(i) Le pair
$\{\mu_1(A),\mu_2(A)\}$ est une base orthonormée
de~$\R^2$ orientée positivement quel que soit
$A\in S^+_*$\\
(ii) $A\cdot\mu_k(A)=\lambda_k(A)\mu_k(A)$\\
(iii) $\mu_k\in\mathcal{C}^\infty(S^+_*;\R^2)$\\
Indication: trouvez une formule explicite pour chaque~$\mu_k(x,y,z)$.


{\bf Exercice 5.}
Est-il possible d'étendre les fonctions~$\mu_k$ sur
tout~$S^+$ de façon qu'elles soient toujours~$\mathcal{C}^\infty$?
Et~$\mathcal{C}^0$ ?

{\bf Exercice 6.}
Étant donnée~$A\in S^+_*$, calculez les (6) dérivées partielles
$\frac{\partial\lambda_1}{\partial x}$,
$\frac{\partial\lambda_1}{\partial y}$,
$\frac{\partial\lambda_1}{\partial z}$,
$\frac{\partial\lambda_2}{\partial x}$,
$\frac{\partial\lambda_2}{\partial y}$,
$\frac{\partial\lambda_2}{\partial z}$
sur le point~$A=(x,y,z)$, directement à partir de la formule de l'exercice 2.

{\bf Exercice 7.}
Même question, mais pour les (12) dérivées partielles concernant les
composantes des~$\mu_k$.

\vfill

Afin de simplifier ces formules et les généraliser à dimension~$N$,
on écrit matriciellement la condition (ii) de l'exercice 4 ainsi:
\begin{equation}\label{eq:spectral}
	A = U^\top \Sigma U
\end{equation}
où~$\Sigma$ est une matrice diagonale avec les valeurs propres, et les colonnes
de~$U$ sont une base orthonormée de vecteurs propres.

{\bf Exercice 8.}
Retrouvez les expressions des exercices 6 et 7, en dérivant la formule
matricielle~(\ref{eq:spectral}) et appliquant des règles formelles de
dérivation (cf.Gilles~2008).

{\bf Exercice 9.}
Généralisez tous ces résultats au cas de dimension~$N$.


\end{document}


% vim:set tw=79 spell spelllang=fr:
